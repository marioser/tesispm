\subsection{Presentación de la empresa}

S\&G Soluciones de Ingeniería S.A.S. es una empresa colombiana clasificada como pequeña empresa según la Ley 905 de 2004 (Ley 590, 2004) y según (Decreto 957 del 05 de junio, 2019) con una planta de personal entre 11 y 50 trabajadores y activos inferiores a 500 SMLV. Fue fundada en marzo de 2019, con el propósito de ofrecer soluciones técnicas de alta calidad en el campo de la automatización industrial, la electricidad, la mecánica, la instrumentación y el control, incursionando además en tecnologías emergentes como el Internet de las Cosas (IoT) y el desarrollo de software a la medida para la industria. \parencite{anifcentrodeestudioseconomicosRetos2021}

S\&G lleva a cabo sus actividades en el dinámico campo de los servicios de ingeniería industrial y construcción de tableros eléctricos, con un enfoque particular en la automatización, instrumentación, control, electricidad y tecnología en desarrollo. El entorno empresarial en el que se desempeña ofrece servicios a sectores industriales de importancia como la alimentaria, minera, farmacéutica, siderúrgica, de impresión y Oil \& Gas, que demandan soluciones tecnológicas cada vez más personalizadas, robustas y sustentables. Estos sectores industriales requieren no solo pericia técnica, sino también una comprensión profunda de sus procesos, exigencias de productividad, seguridad operacional y normativas técnicas específicas. 

La organización evidencia un profundo compromiso en la mejora de sus procesos e implementación de estándares de calidad que favorezcan la consecución de sus metas corporativas, preservando su orientación hacia la optimización de recursos y un aumento en la satisfacción de su clientela.

Para S\&G Soluciones de Ingeniería, el valor real que ofrece la empresa no está simplemente en ejecutar servicios técnicos, sino en resolver problemas por medio de la ingeniería. Más allá de entregar un producto o terminar una tarea, la filosofía de la empresa gira en torno a brindar soluciones que se ajusten de forma precisa a lo que cada cliente necesita, cuidando siempre que haya un equilibrio entre lo técnico, lo económico y lo humano.

S\&G buscar cerrar la brecha debido a la falta de comunicación entre proveedores, la desarticulación de procesos y la sensación de abandono en momentos clave de la ejecución la empresa busca una forma diferente de hacer las cosas, donde el acompañamiento y la comprensión del cliente sean el centro de todo.

Hoy, S\&G se destaca por ir más allá de lo técnico. La empresa se ha propuesto que sus clientes sean aliados estratégicos y el compromiso por encontrar siempre la mejor solución no siempre la más costosa ni la más fácil, sino la más acertada, esto es lo que ha marcado su diferencia y ha permitido que crezca de manera sostenida en un sector exigente.

\subsection{Referentes estratégicos}
\subsubsection{Misión}
Ofrecer soluciones en automatización, instrumentación y eficiencia energética a sus clientes y que conlleven al mejoramiento de la productividad y seguridad en sus procesos, desarrolladas con responsabilidad y excelencia utilizando tecnologías avanzadas que garanticen altos niveles de calidad en sus productos y servicios.

\subsubsection{Visión}
Destacarse en el año 2025 y reconocidos por la calidad de nuestros servicios y posicionamiento técnico, social y económico a nivel nacional y generar un alto grado de confiabilidad, seguridad y absoluta tranquilidad en el manejo los activos de sus clientes.

\subsubsection{Políticas de calidad}
La organización garantizará el cumplimiento tanto de los requisitos legales como de los compromisos adquiridos con el cliente, apoyándose en herramientas como la Matriz Legal, el Manual de Integración de Tableros de Baja Tensión y el Procedimiento de Gestión Interna y de Comunicación de Proyectos, entre otros documentos normativos. Asimismo, se asegurará la calidad en los productos entregados, especialmente en lo relacionado con la integración de tableros eléctricos de baja tensión, mediante la aplicación rigurosa del Procedimiento de Control de Calidad.

\subsubsection{Objetivos de calidad} 
Como objetivos de calidad la empresa S\&G a definido como objetivos de calidad:
\begin{itemize}
    \item Garantizar la calidad de los productos integrados acorde al reglamento
    \item Dar almacenamiento adecuado a todos los materiales conforme a los establecido en los procedimientos de almacenamiento y control de inventarios.
    \item Asegurar la conformidad de los proveedores mediante la evaluación periódica acordes a los procesos de selección y evaluación de proveedores y contratistas.
    \item Identificar los puntos de control y las medidas para garantizar que los tableros, servicios y proyectos cumplan los requisitos legales y de los clientes.
    \item Segurar pruebas de control de calidad para garantizar la conformidad de los productos
    \item Certificar cada producto mediante la trazabilidad de las pruebas ejecutadas.
\end{itemize}


\subsection{Estructura Organizacional}


(ver sistema de gestión)
Detallar los perfiles de cargo o la función de cada uno dentro de la empresa.
Flujo de la información y comunicación, políticas y procedimientos actuales.


\subsection{Productos y servicios ofertados} 
(ver sistema de gestión)

\subsection{Análisis del sector} 
Realizar análisis de PESTEL, FODA y 5 fuerzas de Porter

\subsubsection{Introducción al sector}
Breve presentación de cómo se constituyen los sectores en el mercado más cercano en donde opera S\&G, clasificación del sector (CIIU si aplica).

El sector de la automatización industrial en Colombia ha experimentado un crecimiento significativo en los últimos años, impulsado por la necesidad de las empresas de mejorar su eficiencia operativa y competitividad. Este crecimiento ha sido impulsado por varios factores, incluyendo la adopción de tecnologías avanzadas, la digitalización de procesos y la creciente demanda de soluciones personalizadas en diversas industrias para obtener.

\subsubsection{Tamaño y comportamiento del sector}
Somo ha sido el crecimiento, empleo, inversión, PIB, evoluciones recientes y perspectivas futuras.
