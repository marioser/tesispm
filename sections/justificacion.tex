S\&G Soluciones de Ingeniería, una empresa enfocada en el desarrollo de proyectos de automatización, control, e implementación de soluciones tecnológicas industriales con un enfoque en IoT e industria 4.0. La empresa cuenta con más de 7 años de creación ha logrado anteponerse a las crisis económicas con la producida durante el periodo pandémico del 2019 y la reactivación post pandémica, a su vez, ha experimentado un crecimiento significativo en sus operaciones a nivel regional, nacional e incursiones en el mercado internacional. Sin embargo, debido al crecimiento ha enfrentado retos en la gestión de proyectos, principalmente por la falta de un marco estandarizado. \parencite{castrodiazPropuestaDisenoMetodologia2023}

La ausencia de un sistema claro provoca dificultades concretas, como la pérdida de control sobre los tiempos de entrega, complicaciones en la resolución eficiente de problemas con clientes, una identificación insuficiente de las necesidades específicas en cada proyecto y una administración poco eficiente de los recursos disponibles. Todo esto puede afectar directamente el cierre formal de los proyectos.

En este sentido, en la medida en que aumenta la demanda, los proyectos se vuelven mucho más complejos, los presupuestos mucho más rigurosos para administrar y también los riesgos crecen considerablemente. Trabajar con clientes por fuera de Colombia implica también adaptarse a diferentes culturas administrativas y empresariales, lo que agrega otra capa más de complejidad. La falta de estandarización ha causado errores repetitivos, sobrecostos, retrasos y una potencial percepción negativa por parte de los clientes. @monAlicia-2022

Según la literatura especializada, las empresas inmersas en la Industria 4.0 necesitan mantener procesos eficientes y adaptables para asegurar la satisfacción del cliente y la rentabilidad. Para lograr esto, es fundamental implementar metodologías reconocidas internacionalmente, como las propuestas por Project Management Institute PMI, Australian Institute of Project Management AIPM, ISO21500 e International Project Management Association IPMA entre otros. Estas también ofrecen un marco claro para establecer estrategias y sistematizar la gestión de proyectos, mejorar el uso de los recursos y así poder garantizar entregas a tiempo y dentro del presupuesto (PMI Project Management Institute, 2024); (Kerzner, 2003); (AIPM, 2019); (IPMA, 2024); (ISO, 2024).

Por otra parte, evaluar la madurez organizacional en gestión de proyectos, utilizando modelos como el OPM3® del PMI, permite identificar claramente fortalezas y debilidades actuales permitiendo establecer una ruta de mejora. Planear una implementación para la Oficina de Proyectos (OP) sería un paso estratégico adicional, que facilitaría la estandarización de los procesos y centralizaría el seguimiento de los proyectos.

Por otro lado, según la información proporcionada por (Confecámaras, 2024)las empresas dedicadas a la construcción, actividades profesionales, científicas y técnicas experimentaron un crecimiento del 47,5\% en el año 2023. Este crecimiento implica la creación de al menos un empleo por cada empresa establecida, esto que ratifica la relevancia estratégica del sector técnico y profesional en la economía colombiana.

En relación con lo anterior, estos datos son coherentes a lo planteado por González y Llanes (2024) en el informe Una mirada a las MiPymes en Colombia, en donde afirman que el 99,5\% del tejido empresarial colombiano está compuesto por micro, pequeñas y medianas empresas (MiPymes), las cuales aportan cerca del 40\% al Producto Interno Bruto (PIB) del país.

Según los datos previamente mencionados, implementar una metodología formal de gestión de proyectos y proponer un plan gradual para establecer una Oficina de Proyectos, representaría una estrategia adecuada frente a los desafíos actuales de S\&G Soluciones de Ingeniería. Estas medidas permitirían mejorar directamente la eficiencia, calidad y rentabilidad de los proyectos, creando así una base sólida para un crecimiento sostenido y competitivo dentro del sector tecnológico e industrial. 
