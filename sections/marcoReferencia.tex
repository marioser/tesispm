\setlength{\headheight}{36.09439pt}

Para poder iniciar el proceso de establecimiento de una oficina de proyectos hace indispensable inicialmente establecer un marco conceptual que permita conectar y comprender todos los conceptos, conexiones y actores que participan en el proceso de la gestión de los proyectos, pudiendo permitir una mayor comprensión del alcance, importancia y valor que aporta en todo el ciclo de vida de los proyectos. Teniendo en cuenta lo siguiente y tomando en consideración que el marco teórico establecer las bases para la construcción de todo el proceso al tiempo que establece límites claros abordaremos aspectos generales en cuanto a la gestión de proyectos.

\subsection{Antecedentes y teorías relevantes}
\subsubsection{A nivel nacional}
En el ámbito nacional, la gestión de proyectos ha sido objeto de estudio y desarrollo en diversas instituciones académicas y organizaciones. La Universidad Nacional de Colombia, por ejemplo, ha implementado programas de formación en gestión de proyectos que buscan preparar a los profesionales para enfrentar los desafíos del entorno empresarial actual. Asimismo, el Project Management Institute (PMI) ha promovido la certificación PMP (Project Management Professional) en Colombia, lo que ha contribuido a elevar los estándares de la profesión.

\subsubsection{A nivel internacional}
A nivel internacional, la gestión de proyectos ha evolucionado significativamente en las últimas décadas. Organizaciones como el Project Management Institute (PMI) y la Association for Project Management (APM) han establecido estándares y buenas prácticas que son ampliamente reconocidos en todo el mundo. La publicación del PMBOK (Project Management Body of Knowledge) por parte del PMI ha sido un hito importante en la formalización de la disciplina, proporcionando un marco de referencia para la gestión de proyectos.

\subsection{Gestión de proyectos}
\subsubsection{Proyectos}
La definición de proyecto presentada por los diferentes autores se encuentra directamente relacionadas entre sí, entre las definiciones podemos encontrar “Esfuerzo temporal que se lleva a cabo para crear un producto, servicio o resultado único.” (Project Management Institute, 2021, pág. 4), “Un proyecto también es un proceso y actividades continuas y diarias en las que una organización participa al producir bienes o servicios.” (Pinto, 2020, pág. 25). “Un proyecto es el uso de distintos recursos para realizar una actividad entre los recursos pueden incluirse personas, organizaciones, equipos, materiales e instalaciones” (Gido, 2018).” Es un esfuerzo temporal emprendido para crear un producto, servicio o resultado único” (Larson, 2021, pág. 7). La Asociación Internaciones de Gerencia de Proyectos acuerda “Esfuerzo único y temporal, multidisciplinar y organizado para obtener resultados acordados dentro de los requisitos y limitaciones predefinidos” (International Project Management Association (IPMA®), 2016, pág. 33). Al respecto podemos sintetizar que las definiciones mantienen una relación entre ellas donde finalmente podemos concluir que un proyecto es el esfuerzo o dedicación temporal para ejecutar tareas definidas y relacionadas entre si mediante el uso de los recursos establecidos implementando estrategias para logar creación de un producto o servicio obtener beneficios y logro de objetivos.

\paragraph{Características de un proyecto}
Los proyectos tienen características que los diferencian de otras actividades. Estas características incluyen:

\subsection{Programas}
Un programa es un conjunto de proyectos relacionados entre sí que se gestionan de manera coordinada para obtener beneficios y controlar riesgos que no se pueden gestionar de manera independiente. Los programas permiten a las organizaciones alinear sus proyectos con la estrategia empresarial y optimizar el uso de recursos.

\subsection{Portafolios}
Un portafolio es un conjunto de proyectos y programas que se gestionan de manera conjunta para alcanzar los objetivos estratégicos de una organización. La gestión de portafolios implica la selección, priorización y control de proyectos y programas para maximizar el valor y minimizar el riesgo.

\subsection{Gestión de proyectos}
La gestión de proyectos es la aplicación de conocimientos, habilidades, herramientas y técnicas a las actividades del proyecto para cumplir con los requisitos del mismo. La gestión de proyectos implica la planificación, ejecución y control de proyectos para garantizar que se completen a tiempo, dentro del presupuesto y cumpliendo con los requisitos de calidad.

\subsection{Normas o estándares internacionales de gestión de proyectos}
La gestión de proyectos se basa en normas y estándares internacionales que proporcionan un marco de referencia para la aplicación de buenas prácticas. Estos estándares son desarrollados por organizaciones como el Project Management Institute (PMI) y la International Organization for Standardization (ISO). Algunos de los estándares más reconocidos incluyen:

\subsection{Metodologías de gestión de proyectos}
La metodología de gestión de proyectos es un conjunto de procesos, herramientas y técnicas que se utilizan para gestionar proyectos de manera efectiva. Existen diversas metodologías de gestión de proyectos, cada una con su propio enfoque y metodología. Algunas de las metodologías más reconocidas incluyen:


\subsection{Oficina de gestión de proyectos}
Una oficina de gestión de proyectos (PMO, por sus siglas en inglés) es una unidad organizativa que se encarga de establecer y mantener los estándares de gestión de proyectos dentro de una organización. La PMO proporciona apoyo a los equipos de proyecto, supervisa el desempeño de los proyectos y asegura la alineación con la estrategia empresarial. Existen diferentes tipos de PMOs, que varían en función de su grado de control y apoyo a los proyectos.
\subsubsection{Funciones}
La oficina de gestión de proyectos desempeña diversas funciones, que pueden incluir:
\begin{itemize}
    \item Establecimiento de estándares y metodologías de gestión de proyectos.
    \item Proporcionar formación y desarrollo profesional en gestión de proyectos.
    \item Supervisar el desempeño de los proyectos y programas.
    \item Facilitar la comunicación y colaboración entre los equipos de proyecto.
    \item Gestionar la cartera de proyectos y programas.
    \item Asegurar la alineación de los proyectos con la estrategia empresarial.
    \item Proporcionar apoyo en la gestión de riesgos y problemas.
    \item Realizar auditorías y evaluaciones de proyectos.
    \item Promover la mejora continua en la gestión de proyectos.
\end{itemize}
\subsubsection{Tipos}
La clasificación de las oficinas de gestión de proyectos puede variar según el enfoque y la metodología utilizada. Sin embargo, una clasificación comúnmente aceptada incluye los siguientes tipos:
\begin{itemize}
    \item PMO de soporte: Proporciona apoyo y asesoramiento a los equipos de proyecto, pero no tiene autoridad sobre ellos.
    \item PMO de control: Supervisa y controla el desempeño de los proyectos, asegurando el cumplimiento de los estándares y metodologías.
    \item PMO directiva: Tiene autoridad sobre los proyectos y toma decisiones estratégicas en la gestión de la cartera de proyectos.
    \item PMO consultiva: Actúa como un centro de excelencia en gestión de proyectos, proporcionando asesoramiento y mejores prácticas a la organización.
    \item PMO híbrida: Combina elementos de los diferentes tipos de PMOs, adaptándose a las necesidades específicas de la organización.
\end{itemize}
\subsubsection{Alcance}
El alcance de la oficina de gestión de proyectos puede variar según el tipo de PMO y las necesidades de la organización. Sin embargo, en general, el alcance de una PMO incluye:
\begin{itemize}
    \item Establecimiento de estándares y metodologías de gestión de proyectos.
    \item Proporcionar formación y desarrollo profesional en gestión de proyectos.
    \item Supervisar el desempeño de los proyectos y programas.
    \item Facilitar la comunicación y colaboración entre los equipos de proyecto.
    \item Gestionar la cartera de proyectos y programas.
    \item Asegurar la alineación de los proyectos con la estrategia empresarial.
    \item Proporcionar apoyo en la gestión de riesgos y problemas.
    \item Realizar auditorías y evaluaciones de proyectos.
    \item Promover la mejora continua en la gestión de proyectos.
    \item Fomentar la cultura de gestión de proyectos en la organización.
    \item Proporcionar informes y análisis sobre el desempeño de la cartera de proyectos.
    \item Gestionar la comunicación y el intercambio de información entre los equipos de proyecto.
    \item Facilitar la identificación y gestión de lecciones aprendidas.
    \item Proporcionar apoyo en la gestión de recursos y capacidades.
    \item Promover la colaboración y el trabajo en equipo entre los diferentes proyectos y programas.
    \item Asegurar la alineación de los proyectos con la estrategia empresarial y los objetivos organizacionales.
    \item Proporcionar apoyo en la gestión de cambios y adaptaciones a los proyectos.
    \item Facilitar la identificación y gestión de riesgos y problemas en los proyectos.
\end{itemize}
\subsubsection{Estructura}
\subsubsection{Enfoque}
\subsubsection{Roles}
\subsubsection{Beneficios}
\subsubsection{Desafíos}

\subsection{Gestión de la oficina de proyectos}

\subsection{Madurez de la gestión de proyectos}
La madurez de la gestión de proyectos se refiere al nivel de desarrollo y sofisticación de una organización en la aplicación de prácticas y procesos de gestión de proyectos. A medida que las organizaciones avanzan en su madurez, son capaces de gestionar proyectos de manera más efectiva y eficiente, lo que se traduce en una mayor probabilidad de éxito en la entrega de proyectos.
\subsubsection{Modelos de madurez de la gestión de proyectos}
Los modelos de madurez son herramientas que permiten evaluar y mejorar la capacidad de una organización para gestionar proyectos. Estos modelos proporcionan un marco para identificar áreas de mejora y establecer un plan de acción para avanzar en la madurez organizacional. Existen varios modelos de madurez, cada uno con su propio enfoque y metodología. Algunos de los modelos más reconocidos incluyen:
\subsubsection{Modelo de madurez IPMA}
El modelo de madurez IPMA (International Project Management Association) se centra en la evaluación de la competencia de las organizaciones en la gestión de proyectos. Este modelo se basa en un enfoque de cinco niveles, donde cada nivel representa un grado creciente de madurez en la gestión de proyectos. El modelo IPMA se utiliza para ayudar a las organizaciones a identificar sus fortalezas y debilidades en la gestión de proyectos y a desarrollar un plan de mejora.
\subsubsection{Modelo de madurez CMMI}
El modelo de madurez CMMI (Capability Maturity Model Integration) es un enfoque ampliamente utilizado para evaluar y mejorar los procesos de desarrollo de software y gestión de proyectos. Este modelo se basa en un enfoque de cinco niveles, donde cada nivel representa un grado creciente de madurez en la gestión de proyectos. El modelo CMMI se utiliza para ayudar a las organizaciones a identificar sus fortalezas y debilidades en la gestión de proyectos y a desarrollar un plan de mejora.
\subsubsection{Modelo de madurez OPM3}
El modelo de madurez OPM3 (Organizational Project Management Maturity Model) es un enfoque desarrollado por el Project Management Institute (PMI) para evaluar y mejorar la madurez organizacional en la gestión de proyectos. Este modelo se basa en un enfoque de tres niveles, donde cada nivel representa un grado creciente de madurez en la gestión de proyectos. El modelo OPM3 se utiliza para ayudar a las organizaciones a identificar sus fortalezas y debilidades en la gestión de proyectos y a desarrollar un plan de mejora.
\subsubsection{Modelo de madurez KPMMM}
El modelo de madurez KPMMM (KPMG Project Management Maturity Model) es un enfoque desarrollado por KPMG para evaluar y mejorar la madurez organizacional en la gestión de proyectos. Este modelo se basa en un enfoque de cinco niveles, donde cada nivel representa un grado creciente de madurez en la gestión de proyectos. El modelo KPMMM se utiliza para ayudar a las organizaciones a identificar sus fortalezas y debilidades en la gestión de proyectos y a desarrollar un plan de mejora.
 8987