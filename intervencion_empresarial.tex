\documentclass[12pt]{article}
\usepackage[absolute,overlay]{textpos}
\usepackage{rotating}
\usepackage[utf8]{inputenc}
\usepackage[spanish]{babel}
\usepackage{booktabs}
\usepackage{caption}
\usepackage{pgfplots} % si usas gráficos con TikZ
\pgfplotsset{compat=1.17}

% Cambiar nombres por defecto
\addto\captionsspanish{
  \renewcommand{\tablename}{Tabla}
  \renewcommand{\figurename}{Figura}
}

% Macro simplificada para figuras
\newcommand{\figuraConFuente}[3]{%
  \begin{figure}[H]
    \centering
    \caption{#1}
    \label{fig:#1}
    #2
    \caption*{\small \parbox{\linewidth}{\centering \textbf{Fuente:} #3}}
  \end{figure}
}

% Macro simplificada para tablas
\newcommand{\tablaConFuente}[3]{%
  \begin{table}[H]
    \centering
    \caption{#1}
    \label{tab:#1}
    #2
    \caption*{\small \parbox{\linewidth}{\centering \textbf{Fuente:} #3}}
  \end{table}
}

% Márgenes estándar de 2.54 cm
\usepackage[letterpaper, margin=2.54cm]{geometry}
\usepackage{svg}
% Alineación a la izquierda sin justificar pero con sangría
\usepackage{ragged2e}
\RaggedRight
\setlength{\parindent}{1.5em}  % Sangría de 5 espacios aprox.
\let\raggedsection\centering   % Para mantener títulos centrados
\usepackage{tikz}
\usetikzlibrary{shapes, arrows.meta, positioning}
\usepackage{csquotes}
\usepackage{graphicx}
\usepackage{float}
\usepackage{longtable}
\usepackage{caption}
\usepackage{titlesec}
\usepackage{setspace}
\usepackage{fancyhdr}
\usepackage{tocloft}
\usepackage{float}
\usepackage[hidelinks]{hyperref}
\usepackage[capitalize,noabbrev]{cleveref}
\usepackage{pdflscape}
\usepackage{authblk}
\usepackage[
backend=biber,
style=apa,
sortcites,
url=true
]{biblatex}
\addbibresource{ref.bib}

% Configura nombres personalizados en español
\crefname{figure}{figura}{figuras}
\Crefname{figure}{Figura}{Figuras}
\crefname{table}{tabla}{tablas}
\Crefname{table}{Tabla}{Tablas}

% Formato de títulos de secciones y subsecciones
\titleformat{\section}
  {\normalfont\Large\bfseries\centering}
  {\thesection}{1em}{}

\titleformat{\subsection}
  {\normalfont\large\bfseries}
  {\thesubsection}{1em}{}

\titleformat{\subsubsection}[hang]
  {\normalfont\normalsize\bfseries}
  {\thesubsubsection}{1em}{}

\titleformat{\paragraph}
  {\normalfont\normalsize\itshape}
  {\theparagraph}{1em}{}
\titlespacing*{\paragraph}
  {3.5em}{1.5ex plus .2ex minus .2ex}{.5em}  

\titlespacing*{\subsubsection}{3em}{1.5ex plus 1ex minus .2ex}{.5em}
\addto\captionsspanish{\renewcommand{\tablename}{Tabla}}

% Información general
\title{Propuesta de implementación de metodología y oficina de proyectos \(PO\) para la gestión de proyectos de tecnología y control industrial en la empresa S\&G Soluciones de Ingeniería.}
\author{Mario Javier Serrano Bula}
\date{\today}

% Variables personalizadas
\newcommand{\mydirector}{Nombre completo del director(a)}
\newcommand{\myuniversity}{Universidad EAN}
\newcommand{\myfaculty}{Ingeniería}
\newcommand{\myprogram}{Maestría en Gerencia de Proyectos}
\newcommand{\mycity}{Bogotá, Colombia}
\newcommand{\mytitle}{Propuesta de implementación de metodología y oficina de proyectos \(PO\) para la gestión de proyectos de tecnología y control industrial en la empresa S\&G Soluciones de Ingeniería.}

% Configuración general
%\onehalfspacing  % ← Esta línea activa el interlineado 1.5 en todo el documento
\setstretch{1.5}
\pagestyle{fancy}
\fancyhf{}
\lhead{\footnotesize
  \begin{minipage}[t]{0.8\textwidth}
      \mytitle
  \end{minipage}
}
\rhead{\thepage}
\renewcommand{\cftsecleader}{\cftdotfill{\cftdotsep}}

% Ocultar títulos de listas
\addto\captionsspanish{
  \renewcommand{\listfigurename}{}
  \renewcommand{\listtablename}{}
}

\newcommand{\chapterbreak}{\clearpage \thispagestyle{fancy}}
\setlength{\headsep}{1.5cm}
\setlength{\headheight}{36.1pt}
\setlength{\parskip}{0.75em}  % Espacio vertical entre párrafos

% Permitir numeración y visualización en el índice hasta el nivel 4 (paragraph)
\setcounter{secnumdepth}{4}
\setcounter{tocdepth}{4}

\begin{document}

% Portada
\begin{titlepage}
\makeatletter
\@title\\[3cm]
\@author\\
\makeatother
\vfill
\begin{flushleft}
\myuniversity\\
\myfaculty\\
\myprogram\\
\mycity\\
\today
\end{flushleft}
\end{titlepage}

% Página de presentación
\chapterbreak
\begin{center}
\mytitle\\[2cm]
Trabajo de grado presentado como requisito para optar al título de: Maestría en Gerencia de Proyectos\\[2cm]
Director: \mydirector\\[2cm]
Modalidad: Trabajo Dirigido\\[1.5cm]
\vfill
\myuniversity\\
\myfaculty\\
\myprogram\\
\mycity\\
\today
\end{center}

% Página de aceptación
\chapterbreak
\section*{}
\vspace{4cm}
Firma del jurado\\
Firma del jurado\\
Firma del director del trabajo de grado\\[2cm]
Ciudad, d\'ia/mes/a\~no

% Dedicatoria
\chapterbreak
\section*{Dedicatoria}
\textit{A mis padres por ense\~narme que la exigencia personal tiene sus frutos.}

% Agradecimientos
\chapterbreak
\section*{Agradecimientos}
\paragraph{}
A todas las personas e instituciones que hicieron posible la realizaci\'on de este trabajo.

% Resumen y Abstract
\chapterbreak
\section*{Resumen}
Incluya las ideas principales de su trabajo de grado: tem\'atica, antecedentes, objetivo, metodolog\'ia, resultados y conclusiones.\\
\textbf{Palabras clave:} hasta 7 palabras

\chapterbreak
\section*{Abstract}
Include topic, background, purpose, methodology, results, and conclusions.\\
\textbf{Keywords:} up to 7 words

% Índice
\chapterbreak
\tableofcontents
\newpage
\listoffigures
\listoftables

% CONTENIDO

% Introducción
\chapterbreak
\section{Introducción}
\subsection*{Tema de la intervenci\'on empresarial}
\subsection*{Planteamiento del problema}
\subsection*{Pregunta de investigaci\'on}
\subsection*{Estructura del documento}

% Objetivos
\chapterbreak
\section{Objetivos}
\begin{flushleft}
A medida que las organizaciones incursionan en nuevos mercados, los desafíos asociados a la gestión efectiva de sus proyectos aumentan significativamente. En este contexto, muchas oportunidades pueden derivar en resultados no satisfactorios, debido a la carencia de un enfoque disciplinado y sistemático en la gestión de proyectos. La gestión de proyectos permite anticipar riesgos, mitigar errores y mejorar la toma de decisiones a lo largo del ciclo de vida.
En relación con lo anterior se establecen los siguientes objetivos que guiarán el desarrollo de esta propuesta:

\subsection*{Objetivo general}
Proponer una metodología de gestión de proyectos y una propuesta de implementación para Oficina de Proyectos (OP), para la empresa S\&G Soluciones de Ingeniería, que facilite la evaluación, formulación y planeación, para la ejecución de sus proyectos de ingeniería y tecnología.
\subsection*{Objetivos específicos}
\begin{itemize}
    \item Diagnosticar la situación actual de la gestión de proyectos en S\&G Soluciones de Ingeniería, e identificar su madurez para la gestión de proyectos.
    \item Diseñar una metodología de gestión de proyectos personalizada para S\&G Soluciones de Ingeniería, basada en las mejores prácticas y marcos reconocidos.
    \item Proponer un plan de implementación gradual para la metodología y Oficina de Proyectos (OP) en S\&G Soluciones de Ingeniería, que actúe como un centro de estandarización, soporte y mejora continua para la gestión de proyectos dentro de la organización.
\end{itemize}
\end{flushleft}

% Justificación
\chapterbreak
\section{Justificación}
S\&G Soluciones de Ingeniería, una empresa enfocada en el desarrollo de proyectos de automatización, control, e implementación de soluciones tecnológicas industriales con un enfoque en IoT e industria 4.0. La empresa cuenta con más de 7 años de creación ha logrado anteponerse a las crisis económicas con la producida durante el periodo pandémico del 2019 y la reactivación post pandémica. Entre los años 2022 y 2024 (\Cref{fig:Cantidad de proyectos por país y año}), la empresa S\&G registró una tasa compuesta anual de crecimiento (CAGR) del 63.84\% en la ejecución de proyectos. Este indicador refleja un crecimiento sostenido y progresivo, mostrando una consolidación de su actividad operativa y un crecimiendo en sus operaciones a nivel regional, nacional e incursiones en el mercado internacional . Sin embargo, debido al crecimiento (\Cref{tab:Proyectos registrados por año}), ha enfrentado retos en la gestión de proyectos, principalmente por la falta de un marco estandarizado. 

\figuraConFuente{Cantidad de proyectos por país y año}
{
  \begin{tikzpicture}
    \begin{axis}[
    ybar stacked,
  bar width=30pt,
  enlarge x limits=0.15,
  ymin=0,
  ymax=110,  % ajusta según necesidad
  xlabel={Año},
  ylabel={Cantidad},
  symbolic x coords={2022,2023,2024},
  xtick=data,
  legend style={at={(1.05,1)}, anchor=north west},
  ylabel near ticks,
  xlabel near ticks,
  nodes near coords,
  every node near coord/.append style={
    font=\scriptsize,
    color=black,
    yshift=4pt
  }
  ]

      \addplot+[fill=green!70!black] coordinates {
        (2022,37) (2023,80) (2024,101)
      };
      \addlegendentry{Colombia}

      \addplot+[fill=red!70] coordinates {
        (2022,0) (2023,0) (2024,1)
      };
      \addlegendentry{Guatemala}

      \addplot+[fill=black!70] coordinates {
        (2022,1) (2023,0) (2024,0)
      };
      \addlegendentry{USA}

    \end{axis}
  \end{tikzpicture}
}
{Elaboración propia basada en información interna (2025).}

\tablaConFuente{Proyectos registrados por año}{
  \begin{tabular}{ll}
    \toprule
    \textbf{Año} & \textbf{Total} \\
    \midrule
    2021 & 23 \\
    2022 & 38 \\
    2023 & 80 \\
    2024 & 102 \\
    \midrule
    \textbf{Total general} & \textbf{243} \\
    \bottomrule
  \end{tabular}
}{
  Elaboración propia basada en información interna (2025).
}

 De acuerdo con \parencite{kerzner_using_2019} y \parencite{pmiProcess2023} La ausencia de un sistema claro provoca dificultades concretas, como la pérdida de control, problemas de gestión y complicaciones en la resolución eficiente de problemas con clientes, identificación insuficiente de las necesidades específicas en cada proyecto y administración ineficiente de los recursos disponibles. 

En este sentido y tomando en cuenta a \parencite{wysockiProject2004} en la medida en que aumenta la demanda, los proyectos se vuelven mucho más complejos, los presupuestos mucho más rigurosos para administrar y también los riesgos crecen considerablemente. Teniendo en cuenta lo expuesto por \parencite{juhreManaging2000} Trabajar con clientes por fuera de Colombia implica también adaptarse a diferentes culturas administrativas y empresariales, lo que agrega otra capa más de complejidad. La falta de estandarización ha causado errores repetitivos, sobrecostos, retrasos y una potencial percepción negativa por parte de los clientes. 

Es fundamental implementar metodologías reconocidas internacionalmente, ofrecen un marco claro para establecer estrategias y sistematizar la gestión de proyectos, mejorar el uso de los recursos como también reducir los riesgos sobre tiempo y presupuesto. \parencite{projectmanagementinstitutestandard2021},\parencite{projectmanagementinstituteGuide2017},\parencite{associationforprojectmanagementAPM2019},\parencite{isoISO2021}, Project Management by ICB4 \parencite{hedemanProject2023} y guías como la propuesta por \parencite{kerznerProject2006}

Por otra parte, evaluar la madurez organizacional en gestión de proyectos, utilizando modelos como el OPM3® del PMI \parencite{projectmanagementinstitutestandard2018}, permite identificar claramente fortalezas y debilidades actuales permitiendo establecer una ruta de mejora. Planear una implementación para la Oficina de Proyectos (OP) sería un paso estratégico adicional, que facilitaría la estandarización de los procesos y centralizaría el seguimiento de los proyectos para S\&G.

Por otro lado, en Colombia, según la información proporcionada por \parencite[p. 8]{confecamarasDinamica2025} el subsector de las empresas que desarrollan actividades profesionales, científicas y técnicas, muestran una variación en la contribución de creación de empresas de un 13,6\%, para un total de 2.257 empresas solo en actividades de ingeniería y otras actividades conexas de consultoría para el año 2024. Este crecimiento implica la creación de al menos un empleo por cada empresa establecida, ratificando la relevancia estratégica del sector técnico y profesional en la economía colombiana.

En relación con lo anterior, estos datos son coherentes a lo planteado por \parencite{gonzalezmirada2024} en el informe Una mirada a las MiPymes en Colombia, en donde afirman que el 99,5\% del tejido empresarial colombiano está compuesto por micro, pequeñas y medianas empresas (MiPymes), las cuales aportan cerca del 40\% al Producto Interno Bruto (PIB) del país.

Según los datos previamente mencionados, implementar una metodología formal de gestión de proyectos y proponer un plan gradual para establecer una Oficina de Proyectos, representaría una estrategia adecuada frente a los desafíos actuales de S\&G Soluciones de Ingeniería. Estas medidas permitirían mejorar directamente la eficiencia, calidad y rentabilidad de los proyectos, creando así una base sólida para un crecimiento sostenido y competitivo dentro del sector tecnológico e industrial.

% Marco Institucional
\chapterbreak
\section{Marco Institucional}
\subsection{Presentación de la empresa}

S\&G Soluciones de Ingeniería S.A.S. es una empresa colombiana clasificada como pequeña empresa según la Ley 905 de 2004 (Ley 590, 2004) y según (Decreto 957 del 05 de junio, 2019) con una planta de personal entre 11 y 50 trabajadores y activos inferiores a 500 SMLV. Fue fundada en marzo de 2019, con el propósito de ofrecer soluciones técnicas de alta calidad en el campo de la automatización industrial, la electricidad, la mecánica, la instrumentación y el control, incursionando además en tecnologías emergentes como el Internet de las Cosas (IoT) y el desarrollo de software a la medida para la industria. \parencite{anifcentrodeestudioseconomicosRetos2021}

S\&G lleva a cabo sus actividades en el dinámico campo de los servicios de ingeniería industrial y construcción de tableros eléctricos, con un enfoque particular en la automatización, instrumentación, control, electricidad y tecnología en desarrollo. El entorno empresarial en el que se desempeña ofrece servicios a sectores industriales de importancia como la alimentaria, minera, farmacéutica, siderúrgica, de impresión y Oil \& Gas, que demandan soluciones tecnológicas cada vez más personalizadas, robustas y sustentables. Estos sectores industriales requieren no solo pericia técnica, sino también una comprensión profunda de sus procesos, exigencias de productividad, seguridad operacional y normativas técnicas específicas. 

La organización evidencia un profundo compromiso en la mejora de sus procesos e implementación de estándares de calidad que favorezcan la consecución de sus metas corporativas, preservando su orientación hacia la optimización de recursos y un aumento en la satisfacción de su clientela.

Para S\&G Soluciones de Ingeniería, el valor real que ofrece la empresa no está simplemente en ejecutar servicios técnicos, sino en resolver problemas por medio de la ingeniería. Más allá de entregar un producto o terminar una tarea, la filosofía de la empresa gira en torno a brindar soluciones que se ajusten de forma precisa a lo que cada cliente necesita, cuidando siempre que haya un equilibrio entre lo técnico, lo económico y lo humano.

S\&G buscar cerrar la brecha debido a la falta de comunicación entre proveedores, la desarticulación de procesos y la sensación de abandono en momentos clave de la ejecución la empresa busca una forma diferente de hacer las cosas, donde el acompañamiento y la comprensión del cliente sean el centro de todo.

Hoy, S\&G se destaca por ir más allá de lo técnico. La empresa se ha propuesto que sus clientes sean aliados estratégicos y el compromiso por encontrar siempre la mejor solución no siempre la más costosa ni la más fácil, sino la más acertada, esto es lo que ha marcado su diferencia y ha permitido que crezca de manera sostenida en un sector exigente.

\subsection{Referentes estratégicos}
\subsubsection{Misión}
Ofrecer soluciones en automatización, instrumentación y eficiencia energética a sus clientes y que conlleven al mejoramiento de la productividad y seguridad en sus procesos, desarrolladas con responsabilidad y excelencia utilizando tecnologías avanzadas que garanticen altos niveles de calidad en sus productos y servicios.

\subsubsection{Visión}
Destacarse en el año 2025 y reconocidos por la calidad de nuestros servicios y posicionamiento técnico, social y económico a nivel nacional y generar un alto grado de confiabilidad, seguridad y absoluta tranquilidad en el manejo los activos de sus clientes.

\subsubsection{Políticas de calidad}
La organización garantizará el cumplimiento tanto de los requisitos legales como de los compromisos adquiridos con el cliente, apoyándose en herramientas como la Matriz Legal, el Manual de Integración de Tableros de Baja Tensión y el Procedimiento de Gestión Interna y de Comunicación de Proyectos, entre otros documentos normativos. Asimismo, se asegurará la calidad en los productos entregados, especialmente en lo relacionado con la integración de tableros eléctricos de baja tensión, mediante la aplicación rigurosa del Procedimiento de Control de Calidad.

\subsubsection{Objetivos de calidad} 
Como objetivos de calidad la empresa S\&G a definido como objetivos de calidad:
\begin{itemize}
    \item Garantizar la calidad de los productos integrados acorde al reglamento
    \item Dar almacenamiento adecuado a todos los materiales conforme a los establecido en los procedimientos de almacenamiento y control de inventarios.
    \item Asegurar la conformidad de los proveedores mediante la evaluación periódica acordes a los procesos de selección y evaluación de proveedores y contratistas.
    \item Identificar los puntos de control y las medidas para garantizar que los tableros, servicios y proyectos cumplan los requisitos legales y de los clientes.
    \item Segurar pruebas de control de calidad para garantizar la conformidad de los productos
    \item Certificar cada producto mediante la trazabilidad de las pruebas ejecutadas.
\end{itemize}

\newgeometry{paper=letterpaper, landscape, margin=2.54cm}

\begin{landscape}
\thispagestyle{empty}

% Bloque rotado ajustado a margen izquierdo real del contenido horizontal
\begin{textblock*}{\textheight}(1cm,2cm) % Desde el borde físico de la hoja
  \rotatebox{90}{
    \begin{minipage}[t]{0.85\paperheight}  % Ancho útil horizontal total
      \footnotesize
      \begin{minipage}[t]{0.5\textwidth}
        \RaggedRight \mytitle
      \end{minipage}%
      \begin{minipage}[t]{0.4\textwidth}
        \RaggedLeft \thepage
      \end{minipage}
      %\vspace{10mm}
      \rule{\textwidth}{0.4pt}
    \end{minipage}
  }
\end{textblock*}
\vspace*{0.5px}

\section{Estructura organizacional}
\begin{figure}[H]
\centering
\includesvg[width=0.8\textwidth]{sections/organigramas/organigrama}
\caption{Imagen SVG directa}
\end{figure}

\end{landscape}

\restoregeometry

(ver sistema de gestión)
Detallar los perfiles de cargo o la función de cada uno dentro de la empresa.
Flujo de la información y comunicación, políticas y procedimientos actuales.


\subsection{Productos y servicios ofertados} 
(ver sistema de gestión)

\subsection{Análisis del sector} 
Realizar análisis de PESTEL, FODA y 5 fuerzas de Porter

\subsubsection{Introducción al sector}
Breve presentación de cómo se constituyen los sectores en el mercado más cercano en donde opera S\&G, clasificación del sector (CIIU si aplica).

El sector de la automatización industrial en Colombia ha experimentado un crecimiento significativo en los últimos años, impulsado por la necesidad de las empresas de mejorar su eficiencia operativa y competitividad. Este crecimiento ha sido impulsado por varios factores, incluyendo la adopción de tecnologías avanzadas, la digitalización de procesos y la creciente demanda de soluciones personalizadas en diversas industrias para obtener.

\subsubsection{Tamaño y comportamiento del sector}
Somo ha sido el crecimiento, empleo, inversión, PIB, evoluciones recientes y perspectivas futuras.



% Marco de Referencia
\chapterbreak
\section{Marco de Referencia}
\setlength{\headheight}{36.09439pt}

Para poder iniciar el proceso de establecimiento de una oficina de proyectos hace indispensable inicialmente establecer un marco conceptual que permita conectar y comprender todos los conceptos, conexiones y actores que participan en el proceso de la gestión de los proyectos, pudiendo permitir una mayor comprensión del alcance, importancia y valor que aporta en todo el ciclo de vida de los proyectos. Teniendo en cuenta lo siguiente y tomando en consideración que el marco teórico establecer las bases para la construcción de todo el proceso al tiempo que establece límites claros abordaremos aspectos generales en cuanto a la gestión de proyectos.

\subsection{Antecedentes y teorías relevantes}
\subsubsection{A nivel nacional}
En el ámbito nacional, la gestión de proyectos ha sido objeto de estudio y desarrollo en diversas instituciones académicas y organizaciones. La Universidad Nacional de Colombia, por ejemplo, ha implementado programas de formación en gestión de proyectos que buscan preparar a los profesionales para enfrentar los desafíos del entorno empresarial actual. Asimismo, el Project Management Institute (PMI) ha promovido la certificación PMP (Project Management Professional) en Colombia, lo que ha contribuido a elevar los estándares de la profesión.

\subsubsection{A nivel internacional}
A nivel internacional, la gestión de proyectos ha evolucionado significativamente en las últimas décadas. Organizaciones como el Project Management Institute (PMI) y la Association for Project Management (APM) han establecido estándares y buenas prácticas que son ampliamente reconocidos en todo el mundo. La publicación del PMBOK (Project Management Body of Knowledge) por parte del PMI ha sido un hito importante en la formalización de la disciplina, proporcionando un marco de referencia para la gestión de proyectos.

\subsection{Gestión de proyectos}
\subsubsection{Proyectos}
La definición de proyecto presentada por los diferentes autores se encuentra directamente relacionadas entre sí, entre las definiciones podemos encontrar “Esfuerzo temporal que se lleva a cabo para crear un producto, servicio o resultado único.” (Project Management Institute, 2021, pág. 4), “Un proyecto también es un proceso y actividades continuas y diarias en las que una organización participa al producir bienes o servicios.” (Pinto, 2020, pág. 25). “Un proyecto es el uso de distintos recursos para realizar una actividad entre los recursos pueden incluirse personas, organizaciones, equipos, materiales e instalaciones” (Gido, 2018).” Es un esfuerzo temporal emprendido para crear un producto, servicio o resultado único” (Larson, 2021, pág. 7). La Asociación Internaciones de Gerencia de Proyectos acuerda “Esfuerzo único y temporal, multidisciplinar y organizado para obtener resultados acordados dentro de los requisitos y limitaciones predefinidos” (International Project Management Association (IPMA®), 2016, pág. 33). Al respecto podemos sintetizar que las definiciones mantienen una relación entre ellas donde finalmente podemos concluir que un proyecto es el esfuerzo o dedicación temporal para ejecutar tareas definidas y relacionadas entre si mediante el uso de los recursos establecidos implementando estrategias para logar creación de un producto o servicio obtener beneficios y logro de objetivos.

\paragraph{Características de un proyecto}
Los proyectos tienen características que los diferencian de otras actividades. Estas características incluyen:

\subsection{Programas}
Un programa es un conjunto de proyectos relacionados entre sí que se gestionan de manera coordinada para obtener beneficios y controlar riesgos que no se pueden gestionar de manera independiente. Los programas permiten a las organizaciones alinear sus proyectos con la estrategia empresarial y optimizar el uso de recursos.

\subsection{Portafolios}
Un portafolio es un conjunto de proyectos y programas que se gestionan de manera conjunta para alcanzar los objetivos estratégicos de una organización. La gestión de portafolios implica la selección, priorización y control de proyectos y programas para maximizar el valor y minimizar el riesgo.

\subsection{Gestión de proyectos}
La gestión de proyectos es la aplicación de conocimientos, habilidades, herramientas y técnicas a las actividades del proyecto para cumplir con los requisitos del mismo. La gestión de proyectos implica la planificación, ejecución y control de proyectos para garantizar que se completen a tiempo, dentro del presupuesto y cumpliendo con los requisitos de calidad.

\subsection{Normas o estándares internacionales de gestión de proyectos}
La gestión de proyectos se basa en normas y estándares internacionales que proporcionan un marco de referencia para la aplicación de buenas prácticas. Estos estándares son desarrollados por organizaciones como el Project Management Institute (PMI) y la International Organization for Standardization (ISO). Algunos de los estándares más reconocidos incluyen:

\subsection{Metodologías de gestión de proyectos}
La metodología de gestión de proyectos es un conjunto de procesos, herramientas y técnicas que se utilizan para gestionar proyectos de manera efectiva. Existen diversas metodologías de gestión de proyectos, cada una con su propio enfoque y metodología. Algunas de las metodologías más reconocidas incluyen:


\subsection{Oficina de gestión de proyectos}
Una oficina de gestión de proyectos (PMO, por sus siglas en inglés) es una unidad organizativa que se encarga de establecer y mantener los estándares de gestión de proyectos dentro de una organización. La PMO proporciona apoyo a los equipos de proyecto, supervisa el desempeño de los proyectos y asegura la alineación con la estrategia empresarial. Existen diferentes tipos de PMOs, que varían en función de su grado de control y apoyo a los proyectos.
\subsubsection{Funciones}
La oficina de gestión de proyectos desempeña diversas funciones, que pueden incluir:
\begin{itemize}
    \item Establecimiento de estándares y metodologías de gestión de proyectos.
    \item Proporcionar formación y desarrollo profesional en gestión de proyectos.
    \item Supervisar el desempeño de los proyectos y programas.
    \item Facilitar la comunicación y colaboración entre los equipos de proyecto.
    \item Gestionar la cartera de proyectos y programas.
    \item Asegurar la alineación de los proyectos con la estrategia empresarial.
    \item Proporcionar apoyo en la gestión de riesgos y problemas.
    \item Realizar auditorías y evaluaciones de proyectos.
    \item Promover la mejora continua en la gestión de proyectos.
\end{itemize}
\subsubsection{Tipos}
La clasificación de las oficinas de gestión de proyectos puede variar según el enfoque y la metodología utilizada. Sin embargo, una clasificación comúnmente aceptada incluye los siguientes tipos:
\begin{itemize}
    \item PMO de soporte: Proporciona apoyo y asesoramiento a los equipos de proyecto, pero no tiene autoridad sobre ellos.
    \item PMO de control: Supervisa y controla el desempeño de los proyectos, asegurando el cumplimiento de los estándares y metodologías.
    \item PMO directiva: Tiene autoridad sobre los proyectos y toma decisiones estratégicas en la gestión de la cartera de proyectos.
    \item PMO consultiva: Actúa como un centro de excelencia en gestión de proyectos, proporcionando asesoramiento y mejores prácticas a la organización.
    \item PMO híbrida: Combina elementos de los diferentes tipos de PMOs, adaptándose a las necesidades específicas de la organización.
\end{itemize}
\subsubsection{Alcance}
El alcance de la oficina de gestión de proyectos puede variar según el tipo de PMO y las necesidades de la organización. Sin embargo, en general, el alcance de una PMO incluye:
\begin{itemize}
    \item Establecimiento de estándares y metodologías de gestión de proyectos.
    \item Proporcionar formación y desarrollo profesional en gestión de proyectos.
    \item Supervisar el desempeño de los proyectos y programas.
    \item Facilitar la comunicación y colaboración entre los equipos de proyecto.
    \item Gestionar la cartera de proyectos y programas.
    \item Asegurar la alineación de los proyectos con la estrategia empresarial.
    \item Proporcionar apoyo en la gestión de riesgos y problemas.
    \item Realizar auditorías y evaluaciones de proyectos.
    \item Promover la mejora continua en la gestión de proyectos.
    \item Fomentar la cultura de gestión de proyectos en la organización.
    \item Proporcionar informes y análisis sobre el desempeño de la cartera de proyectos.
    \item Gestionar la comunicación y el intercambio de información entre los equipos de proyecto.
    \item Facilitar la identificación y gestión de lecciones aprendidas.
    \item Proporcionar apoyo en la gestión de recursos y capacidades.
    \item Promover la colaboración y el trabajo en equipo entre los diferentes proyectos y programas.
    \item Asegurar la alineación de los proyectos con la estrategia empresarial y los objetivos organizacionales.
    \item Proporcionar apoyo en la gestión de cambios y adaptaciones a los proyectos.
    \item Facilitar la identificación y gestión de riesgos y problemas en los proyectos.
\end{itemize}
\subsubsection{Estructura}
\subsubsection{Enfoque}
\subsubsection{Roles}
\subsubsection{Beneficios}
\subsubsection{Desafíos}

\subsection{Gestión de la oficina de proyectos}

\subsection{Madurez de la gestión de proyectos}
La madurez de la gestión de proyectos se refiere al nivel de desarrollo y sofisticación de una organización en la aplicación de prácticas y procesos de gestión de proyectos. A medida que las organizaciones avanzan en su madurez, son capaces de gestionar proyectos de manera más efectiva y eficiente, lo que se traduce en una mayor probabilidad de éxito en la entrega de proyectos.
\subsubsection{Modelos de madurez de la gestión de proyectos}
Los modelos de madurez son herramientas que permiten evaluar y mejorar la capacidad de una organización para gestionar proyectos. Estos modelos proporcionan un marco para identificar áreas de mejora y establecer un plan de acción para avanzar en la madurez organizacional. Existen varios modelos de madurez, cada uno con su propio enfoque y metodología. Algunos de los modelos más reconocidos incluyen:
\subsubsection{Modelo de madurez IPMA}
El modelo de madurez IPMA (International Project Management Association) se centra en la evaluación de la competencia de las organizaciones en la gestión de proyectos. Este modelo se basa en un enfoque de cinco niveles, donde cada nivel representa un grado creciente de madurez en la gestión de proyectos. El modelo IPMA se utiliza para ayudar a las organizaciones a identificar sus fortalezas y debilidades en la gestión de proyectos y a desarrollar un plan de mejora.
\subsubsection{Modelo de madurez CMMI}
El modelo de madurez CMMI (Capability Maturity Model Integration) es un enfoque ampliamente utilizado para evaluar y mejorar los procesos de desarrollo de software y gestión de proyectos. Este modelo se basa en un enfoque de cinco niveles, donde cada nivel representa un grado creciente de madurez en la gestión de proyectos. El modelo CMMI se utiliza para ayudar a las organizaciones a identificar sus fortalezas y debilidades en la gestión de proyectos y a desarrollar un plan de mejora.
\subsubsection{Modelo de madurez OPM3}
El modelo de madurez OPM3 (Organizational Project Management Maturity Model) es un enfoque desarrollado por el Project Management Institute (PMI) para evaluar y mejorar la madurez organizacional en la gestión de proyectos. Este modelo se basa en un enfoque de tres niveles, donde cada nivel representa un grado creciente de madurez en la gestión de proyectos. El modelo OPM3 se utiliza para ayudar a las organizaciones a identificar sus fortalezas y debilidades en la gestión de proyectos y a desarrollar un plan de mejora.
\subsubsection{Modelo de madurez KPMMM}
El modelo de madurez KPMMM (KPMG Project Management Maturity Model) es un enfoque desarrollado por KPMG para evaluar y mejorar la madurez organizacional en la gestión de proyectos. Este modelo se basa en un enfoque de cinco niveles, donde cada nivel representa un grado creciente de madurez en la gestión de proyectos. El modelo KPMMM se utiliza para ayudar a las organizaciones a identificar sus fortalezas y debilidades en la gestión de proyectos y a desarrollar un plan de mejora.
 8987

% Diseño Metodológico
\chapterbreak
\section{Diseño Metodológico}
Describir el enfoque, fases, metodología de diagnóstico y propuesta de intervención.

\begin{table}[H]
\begin{tabular}{|l|c|c|c|c|}
\hline
\textbf{Proyecto} & \textbf{Duración (meses)} & \textbf{Costo (USD)} & \textbf{Estado} & \textbf{Responsable} \\
\hline
Automatización Planta A & 6 & 45,000 & Finalizado & J. Martínez \\
Implementación CRM & 4 & 18,200 & En curso & L. Rodríguez \\
Migración a la nube & 8 & 63,750 & Pendiente & M. González \\
Actualización SCADA & 5 & 29,500 & En curso & A. Restrepo \\
Capacitación IoT & 2 & 9,800 & Finalizado & C. Fernández \\
\hline
\end{tabular}
\centering
\caption{Resumen de desempeño de proyectos}
\label{tab:resumen_proyectos}
\end{table}

% Diagnóstico Organizacional
\chapterbreak
\section{Diagnóstico Organizacional}
\subsection{Procesamiento estadístico de datos}
\subsection{Análisis de resultados}

% Plan de Intervención
\chapterbreak
\section{Plan de Intervención}
Define un plan de intervención para la empresa, con base en los resultados del diagnóstico organizacional.

% Conclusiones y Recomendaciones
\chapterbreak
\section{Conclusiones y Recomendaciones}
\subsection{Conclusiones}
\subsection{Recomendaciones}

% Referencias
\chapterbreak
\def\refname{}
\section*{Referencias}
\printbibliography[heading=bibintoc]

% Anexos
\chapterbreak
\appendix
\section*{Anexo A. Nombre del anexo}
Contenido del anexo en esta sección, puede ser reconocido de forma automática. Con cambios de formato, por ejemplo, si se desea incluir una tabla o figura, se puede hacer de la siguiente manera:

\end{document}
