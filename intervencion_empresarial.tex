\documentclass[12pt]{article}
\usepackage[utf8]{inputenc}
\usepackage[spanish]{babel}

% Márgenes estándar de 2.54 cm
\usepackage[letterpaper, margin=2.54cm]{geometry}

% Alineación a la izquierda sin justificar pero con sangría
\usepackage{ragged2e}
\RaggedRight
\setlength{\parindent}{1.5em}  % Sangría de 5 espacios aprox.
\let\raggedsection\centering   % Para mantener títulos centrados

\usepackage{csquotes}
\usepackage{graphicx}
\usepackage{float}
\usepackage{longtable}
\usepackage{caption}
\usepackage{titlesec}
\usepackage{setspace}
\usepackage{fancyhdr}
\usepackage{tocloft}
\usepackage[hidelinks]{hyperref}
\usepackage{pdflscape}
\usepackage{authblk}
\usepackage[
backend=biber,
style=apa,
sortcites,
url=true
]{biblatex}
\addbibresource{intervencion_empresarial.bib}

% Formato de títulos de secciones y subsecciones
\titleformat{\section}
  {\normalfont\Large\bfseries\centering}
  {\thesection}{1em}{}

\titleformat{\subsection}
  {\normalfont\large\bfseries}
  {\thesubsection}{1em}{}

\titleformat{\subsubsection}
  {\normalfont\normalsize\bfseries}
  {\thesubsubsection}{1em}{}

% Información general
\title{Propuesta de implementación de metodología y oficina de proyectos \(PO\) para la gestión de proyectos de tecnología y control industrial en la empresa S\&G Soluciones de Ingeniería.}
\author{Mario Javier Serrano Bula}
\date{\today}

% Variables personalizadas
\newcommand{\mydirector}{Nombre completo del director(a)}
\newcommand{\myuniversity}{Universidad EAN}
\newcommand{\myfaculty}{Ingeniería}
\newcommand{\myprogram}{Maestría en Gerencia de Proyectos}
\newcommand{\mycity}{Bogotá, Colombia}
\newcommand{\mytitle}{Propuesta de implementación de metodología y oficina de proyectos \(PO\) para la gestión de proyectos de tecnología y control industrial en la empresa S\&G Soluciones de Ingeniería.}

% Configuración general
\onehalfspacing  % ← Esta línea activa el interlineado 1.5 en todo el documento
\pagestyle{fancy}
\fancyhf{}
\lhead{\footnotesize
  \begin{minipage}[t]{0.8\textwidth}
      \mytitle
  \end{minipage}
}
\rhead{\thepage}
\renewcommand{\cftsecleader}{\cftdotfill{\cftdotsep}}

% Ocultar títulos de listas
\addto\captionsspanish{
  \renewcommand{\listfigurename}{}
  \renewcommand{\listtablename}{}
}

\newcommand{\chapterbreak}{\clearpage \thispagestyle{fancy}}
\setlength{\headsep}{1.5cm}

\begin{document}

% Portada
\begin{titlepage}
\makeatletter
\@title\\[3cm]
\@author\\
\makeatother
\vfill
\begin{flushleft}
\myuniversity\\
\myfaculty\\
\myprogram\\
\mycity\\
\today
\end{flushleft}
\end{titlepage}

% Página de presentación
\chapterbreak
\begin{center}
\mytitle\\[2cm]
Trabajo de grado presentado como requisito para optar al título de: Magíster en Gerencia de Proyectos\\[2cm]
Director: \mydirector\\[2cm]
Modalidad: Trabajo Dirigido\\[1.5cm]
\vfill
\myuniversity\\
\myfaculty\\
\myprogram\\
\mycity\\
\today
\end{center}

% Página de aceptación
\chapterbreak
\section*{}
 \noindent
\begin{minipage}[t]{0.3\textwidth}
\textbf{Nota de aceptación:}

\end{minipage}
\hfill
\begin{minipage}[t]{0.6\textwidth}

\vspace{0.8cm}
\rule{\textwidth}{0.4pt}\\[0.5cm]
\rule{\textwidth}{0.4pt}\\[0.5cm]
\rule{\textwidth}{0.4pt}\\[0.5cm]
\rule{\textwidth}{0.4pt}\\[0.5cm]
\rule{\textwidth}{0.4pt}\\[2cm]
\rule{0.9\textwidth}{0.4pt}\\
\raggedright Firma del jurado

\vspace{1.2cm}
\rule{0.9\textwidth}{0.4pt}\\
\raggedright Firma del jurado

\vspace{1.2cm}
\rule{0.9\textwidth}{0.4pt}\\
\raggedright Firma del director del trabajo de grado
\end{minipage}

\vspace{2cm}
\vfill
Ciudad, \today

% Dedicatoria
\chapterbreak
\section*{Dedicatoria}
\textit{A mis padres por ense\~narme que la exigencia personal tiene sus frutos.}

% Agradecimientos
\chapterbreak
\section*{Agradecimientos}
\paragraph{}
A todas las personas e instituciones que hicieron posible la realizaci\'on de este trabajo.

% Resumen y Abstract
\chapterbreak
\section*{Resumen}
Incluya las ideas principales de su trabajo de grado: tem\'atica, antecedentes, objetivo, metodolog\'ia, resultados y conclusiones.\\
\textbf{Palabras clave:} hasta 7 palabras

\chapterbreak
\section*{Abstract}
Include topic, background, purpose, methodology, results, and conclusions.\\
\textbf{Keywords:} up to 7 words

% Índice
\chapterbreak
\tableofcontents
\newpage
\listoffigures
\listoftables

% CONTENIDO

% Introducción
\chapterbreak
\section{Introducción}
\subsection*{Tema de la intervenci\'on empresarial}
\subsection*{Planteamiento del problema}
\subsection*{Pregunta de investigaci\'on}
\subsection*{Estructura del documento}

% Objetivos
\chapterbreak
\section{Objetivos}
\begin{flushleft}
A medida que las organizaciones incursionan en nuevos mercados, los desafíos asociados a la gestión efectiva de sus proyectos aumentan significativamente. En este contexto, muchas oportunidades pueden derivar en resultados no satisfactorios debido a la carencia de un enfoque disciplinado y sistemático en la gestión de proyectos. En contraposición, una adecuada gestión de proyectos dota a las organizaciones de herramientas metodológicas que permiten anticipar riesgos, mitigar errores y mejorar la toma de decisiones a lo largo del ciclo de vida de los proyectos.
Con el fin de alcanzar una gestión de excelencia, se establecen los siguientes objetivos que guiarán el desarrollo de esta propuesta:

\subsection*{Objetivo general}
Proponer una metodología de gestión de proyectos, junto con el diseño de una Oficina de Proyectos (OP), para la empresa S\&G Soluciones de Ingeniería, que facilite la evaluación, formulación y análisis de la planeación, y contribuya a mejorar la eficiencia y optimización en la ejecución de sus proyectos de tecnología y servicios de ingeniería en el sector industrial. 

\subsection*{Objetivos específicos}
\begin{itemize}
    \item Diagnosticar la situación actual de la gestión de proyectos en S\&G Soluciones de Ingeniería, e identificar su madurez para la gestión de proyectos.
    \item Analizar las brechas existentes entre la situación actual de la gestión de proyectos y las condiciones requeridas para adoptar la metodología propuesta.
    \item Diseñar una metodología de gestión de proyectos personalizada para S\&G Soluciones de Ingeniería, basada en las mejores prácticas y marcos reconocidos.
    \item Proponer un plan de implementación gradual para la metodología y Oficina de Proyectos (OP) en S\&G Soluciones de Ingeniería, que actúe como un centro de estandarización, soporte y mejora continua para la gestión de proyectos dentro de la organización.
\end{itemize}
\end{flushleft}

% Justificación
\chapterbreak
\section{Justificación}
S\&G Soluciones de Ingeniería, una empresa enfocada en el desarrollo de proyectos de automatización, control, e implementación de soluciones tecnológicas industriales con un enfoque en IoT e industria 4.0. La empresa cuenta con más de 7 años de creación ha logrado anteponerse a las crisis económicas con la producida durante el periodo pandémico del 2019 y la reactivación post pandémica, a su vez, ha experimentado un crecimiento significativo en sus operaciones a nivel regional, nacional e incursiones en el mercado internacional. Sin embargo, debido al crecimiento ha enfrentado retos en la gestión de proyectos, principalmente por la falta de un marco estandarizado. \parencite{castrodiazPropuestaDisenoMetodologia2023}

La ausencia de un sistema claro provoca dificultades concretas, como la pérdida de control sobre los tiempos de entrega, complicaciones en la resolución eficiente de problemas con clientes, una identificación insuficiente de las necesidades específicas en cada proyecto y una administración poco eficiente de los recursos disponibles. Todo esto puede afectar directamente el cierre formal de los proyectos.

En este sentido, en la medida en que aumenta la demanda, los proyectos se vuelven mucho más complejos, los presupuestos mucho más rigurosos para administrar y también los riesgos crecen considerablemente. Trabajar con clientes por fuera de Colombia implica también adaptarse a diferentes culturas administrativas y empresariales, lo que agrega otra capa más de complejidad. La falta de estandarización ha causado errores repetitivos, sobrecostos, retrasos y una potencial percepción negativa por parte de los clientes. @monAlicia-2022

Según la literatura especializada, las empresas inmersas en la Industria 4.0 necesitan mantener procesos eficientes y adaptables para asegurar la satisfacción del cliente y la rentabilidad. Para lograr esto, es fundamental implementar metodologías reconocidas internacionalmente, como las propuestas por Project Management Institute PMI, Australian Institute of Project Management AIPM, ISO21500 e International Project Management Association IPMA entre otros. Estas también ofrecen un marco claro para establecer estrategias y sistematizar la gestión de proyectos, mejorar el uso de los recursos y así poder garantizar entregas a tiempo y dentro del presupuesto (PMI Project Management Institute, 2024); (Kerzner, 2003); (AIPM, 2019); (IPMA, 2024); (ISO, 2024).

Por otra parte, evaluar la madurez organizacional en gestión de proyectos, utilizando modelos como el OPM3® del PMI, permite identificar claramente fortalezas y debilidades actuales permitiendo establecer una ruta de mejora. Planear una implementación para la Oficina de Proyectos (OP) sería un paso estratégico adicional, que facilitaría la estandarización de los procesos y centralizaría el seguimiento de los proyectos.

Por otro lado, según la información proporcionada por (Confecámaras, 2024)las empresas dedicadas a la construcción, actividades profesionales, científicas y técnicas experimentaron un crecimiento del 47,5\% en el año 2023. Este crecimiento implica la creación de al menos un empleo por cada empresa establecida, esto que ratifica la relevancia estratégica del sector técnico y profesional en la economía colombiana.

En relación con lo anterior, estos datos son coherentes a lo planteado por González y Llanes (2024) en el informe Una mirada a las MiPymes en Colombia, en donde afirman que el 99,5\% del tejido empresarial colombiano está compuesto por micro, pequeñas y medianas empresas (MiPymes), las cuales aportan cerca del 40\% al Producto Interno Bruto (PIB) del país.

Según los datos previamente mencionados, implementar una metodología formal de gestión de proyectos y proponer un plan gradual para establecer una Oficina de Proyectos, representaría una estrategia adecuada frente a los desafíos actuales de S\&G Soluciones de Ingeniería. Estas medidas permitirían mejorar directamente la eficiencia, calidad y rentabilidad de los proyectos, creando así una base sólida para un crecimiento sostenido y competitivo dentro del sector tecnológico e industrial. 


% Marco Institucional
\chapterbreak
\section{Marco Institucional}
Descripción de la organización, su misión, visión, estructura, productos/servicios, sector económico, etc.

% Marco de Referencia
\chapterbreak
\section{Marco de Referencia}
\subsection*{Antecedentes y teorías relevantes}
\subsection*{Figuras}
\begin{figure}[H]
\centering
\includegraphics[width=0.7\textwidth]{assets/figura1.png}
\caption{Esquema de fuerzas y su relación. Fuente: adaptado de Hawking (2010)}
\end{figure}

\subsection*{Tablas}
\begin{table}[H]
\centering
\caption{Participación de las energías renovables primarias}
\begin{tabular}{|l|l|l|}
\hline
Región & Energías renovables & Biomasa \\
\hline
Latinoamérica & 28.9\% & 62.4\% \\
Colombia & 27.7\% & 54.4\% \\
Alemania & 3.8\% & 65.8\% \\
Mundial & 13.1\% & 79.4\% \\
\hline
\end{tabular}
\end{table}

% Diseño Metodológico
\chapterbreak
\section{Diseño Metodológico}
Describir el enfoque, fases, metodología de diagnóstico \cite{gonzalezModeloAdministracionProyectos2012} y propuesta de intervención.

% Página horizontal con tabla amplia
\clearpage
\thispagestyle{empty}
\begin{landscape}
\begin{table}[H]
\centering
\caption{Comparativo de tecnologías por país}
\begin{tabular}{|l|c|c|c|c|c|c|}
\hline
País & Solar (MW) & Eólica (MW) & Biomasa (MW) & Hidro (MW) & Geotérmica (MW) & Total Renovables (MW) \\
\hline
Colombia & 100 & 200 & 50 & 1100 & 0 & 1450 \\
Brasil & 500 & 300 & 100 & 2000 & 50 & 2950 \\
Chile & 400 & 450 & 80 & 1200 & 30 & 2160 \\
México & 600 & 700 & 120 & 1800 & 100 & 3320 \\
Perú & 250 & 300 & 60 & 1500 & 20 & 2130 \\
\hline
\end{tabular}
\end{table}
\end{landscape}

% Diagnóstico Organizacional
\chapterbreak
\section{Diagnóstico Organizacional}
\subsection*{Procesamiento estadístico de datos}
\subsection*{Análisis de resultados}

% Plan de Intervención
\chapterbreak
\section{Plan de Intervención}
Describir la propuesta de mejora/intervención a implementar.

% Conclusiones y Recomendaciones
\chapterbreak
\section{Conclusiones y Recomendaciones}
\subsection*{Conclusiones}
\subsection*{Recomendaciones}

% Referencias
\chapterbreak
\section*{Referencias}
\printbibliography[heading=bibintoc]

% Anexos
\chapterbreak
\appendix
\section*{Anexo A. Nombre del anexo}
Contenido del anexo en esta sección, puede ser reconocido de forma automática

\end{document}